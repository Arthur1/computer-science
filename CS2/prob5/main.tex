\documentclass{jsarticle}
\usepackage[dvipdfmx]{graphicx}
\usepackage{listings,jlisting}

\title{コンピュータサイエンス第二\\課題5「実数」}

\author{学籍番号******** ****}
\date{}
\begin{document}
\maketitle

\section{課題(a)}

\subsection{実行結果}

\begin{lstlisting}[language=sh]
(2.0)^50 - 1.0 = 1125899906842623.00
(2.0)^50       = 1125899906842624.00
(2.0)^50 + 1.0 = 1125899906842625.00
(2.0)^50 + 1.5 = 1125899906842625.50
(2.0)^50 + 2.0 = 1125899906842626.00

(2.0)^51 - 1.0 = 2251799813685247.00
(2.0)^51       = 2251799813685248.00
(2.0)^51 + 1.0 = 2251799813685249.00
(2.0)^51 + 1.5 = 2251799813685249.50
(2.0)^51 + 2.0 = 2251799813685250.00

(2.0)^52 - 1.0 = 4503599627370495.00
(2.0)^52       = 4503599627370496.00
(2.0)^52 + 1.0 = 4503599627370497.00
(2.0)^52 + 1.5 = 4503599627370498.00
(2.0)^52 + 2.0 = 4503599627370498.00

(2.0)^53 - 1.0 = 9007199254740991.00
(2.0)^53       = 9007199254740992.00
(2.0)^53 + 1.0 = 9007199254740992.00
(2.0)^53 + 1.5 = 9007199254740994.00
(2.0)^53 + 2.0 = 9007199254740994.00

(2.0)^54 - 1.0 = 18014398509481984.00
(2.0)^54       = 18014398509481984.00
(2.0)^54 + 1.0 = 18014398509481984.00
(2.0)^54 + 1.5 = 18014398509481984.00
(2.0)^54 + 2.0 = 18014398509481984.00
\end{lstlisting}

\subsection{考察}

$i$の値が52以上になると、表示される数字が数学的に正しくない値となる。これは、計算結果の数が大きくなり、借数52ビットで表せる量を超えてしまったため、丸められ誤差が発生したからである。倍精度浮動小数点数では、$2^{52}$から$2^{53}$の数は整数となり、$2^{52}$から$2^{53}$の数は偶数となる。

\section{課題(b)}

\subsection{真のネイピア数と等しいか}

実数の仮数は52bit分しか持てないため、無理数を表現することはできない。ネイピア数は無理数であるから、真のネイピア数ではないということが分かる。本プログラムで表示される値は、真のネイピア数をプログラムにおける実数で表せるように丸めたものであると言える。

\section{課題(c)}

\subsection{1番目の方法}

1番目の式をプログラムにしたものがprog3-1.rbである。実行結果は以下の通りである。

\begin{lstlisting}[language=sh]
$ ruby prog3-1.rb
g(1)= 2.000000000000000000000000000000000000000000000000000000000000
g(2)= 2.500000000000000000000000000000000000000000000000000000000000
g(3)= 2.666666666666666518636930049979127943515777587890625000000000
g(4)= 2.708333333333333037273860099958255887031555175781250000000000
g(5)= 2.716666666666666341001246109954081475734710693359375000000000
g(6)= 2.718055555555555447000415369984693825244903564453125000000000
g(7)= 2.718253968253968366752815200015902519226074218750000000000000
g(8)= 2.718278769841270037233016410027630627155303955078125000000000
g(9)= 2.718281525573192247691167722223326563835144042968750000000000
g(10)= 2.718281801146384513145903838449157774448394775390625000000000
g(11)= 2.718281826198492900914516212651506066322326660156250000000000
g(12)= 2.718281828286168710917536373017355799674987792968750000000000
g(13)= 2.718281828446759362805096316151320934295654296875000000000000
g(14)= 2.718281828458230187095523433526977896690368652343750000000000
g(15)= 2.718281828458994908714885241352021694183349609375000000000000
g(16)= 2.718281828459042870349549048114567995071411132812500000000000
g(17)= 2.718281828459045534884808148490265011787414550781250000000000
g(18)= 2.718281828459045534884808148490265011787414550781250000000000
\end{lstlisting}

$n$の値が18のときはじめて、計算結果が変わらなくなる。

\subsection{2番目の方法}

2番目の式をプログラムにしたものがprog3-2.rbである。実行結果は以下の通りである。

\begin{lstlisting}[language=sh]
$ ruby prog3-2.rb
g(1)= 2.000000000000000000000000000000000000000000000000000000000000
g(2)= 2.500000000000000000000000000000000000000000000000000000000000
g(3)= 2.666666666666666518636930049979127943515777587890625000000000
g(4)= 2.708333333333333037273860099958255887031555175781250000000000
g(5)= 2.716666666666666785090455960016697645187377929687500000000000
g(6)= 2.718055555555555447000415369984693825244903564453125000000000
g(7)= 2.718253968253968366752815200015902519226074218750000000000000
g(8)= 2.718278769841269593143806559965014457702636718750000000000000
g(9)= 2.718281525573192247691167722223326563835144042968750000000000
g(10)= 2.718281801146384513145903838449157774448394775390625000000000
g(11)= 2.718281826198492900914516212651506066322326660156250000000000
g(12)= 2.718281828286168710917536373017355799674987792968750000000000
g(13)= 2.718281828446758918715886466088704764842987060546875000000000
g(14)= 2.718281828458229743006313583464361727237701416015625000000000
g(15)= 2.718281828458994464625675391289405524730682373046875000000000
g(16)= 2.718281828459041982171129347989335656166076660156250000000000
g(17)= 2.718281828459045090795598298427648842334747314453125000000000
g(18)= 2.718281828459045090795598298427648842334747314453125000000000
\end{lstlisting}

こちらも、$n$の値が18のときはじめて、計算結果が変わらなくなる。

\subsection{2つの値の違い}

真の値はこの2つの計算結果の中間となる。これは、実数として丸められた値をさらに計算に使用し、それを繰り返しているため、加法の順序が異なると最終的な値も変わってきてしまうからである。

\end{document}